\newcommand{\layersInRealLife} {
  \frametitle{Уровни (levels / layers) в реальной жизни}
  \framesubtitle{Уровни характеризуются \textbf{обязанностями} и определяют \textbf{кто над кем главнее (выше)}}

  \begin{itemize}
    \item Директор конторы (высокий уровень)
      \begin{itemize}
        \item Художник (низкий уровень)
        \item Программист (низкий уровень)
        \begin{itemize}
          \item Младший программист (самый низкий уровень)
        \end{itemize}
      \end{itemize}
  \end{itemize}
}

\newcommand{\paradigmsImpDecl} {
  \node (imper) [bigrect, xshift=-1cm] {Императивное};
  \node (decl) [bigrect, xshift=5cm, yshift=2.5cm] {Декларативное};
}

\newcommand{\paradigmsAlg} {
  \paradigmsImpDecl

  \node (alg) [process, dashed, xshift=-1cm, yshift=1cm] {Алгоритмическое};
    \node [xshift=-1cm, yshift=0.65cm] {Блок-схемы, словесное описание};
}

\newcommand{\paradigmsStruct} {
  \paradigmsAlg

  \node (struct) [process, xshift=0cm, yshift=0cm] {Структурное};
    \node [xshift=0cm, yshift=-0.35cm] {Ограничение алгоритмического};
  \draw [arrow] (alg) -- (struct);
}

\newcommand{\paradigmsProc} {
  \paradigmsStruct

  \node (proc) [process, xshift=-1cm, yshift=-1cm] {Процедурное};
    \node [xshift=-1cm, yshift=-1.35cm] {C, Pascal...};
  \draw [arrow] (struct) -- (proc);
}

\newcommand{\paradigmsOop} {
  \paradigmsProc

  \node (oop) [process, xshift=-1.5cm, yshift=-2cm] {Объектно-ориентированное (ООП)};
    \node [xshift=-1.5cm, yshift=-2.35cm] {C++, C\#, Java, Python, Ruby, JS, ...};
  \draw [arrow] (proc) -- (oop);
}

\newcommand{\paradigmsDecls} {
  \paradigmsOop

  \node (notprog) [process, dashed, xshift=5cm, yshift=4cm] {Описания};
    \node [xshift=5cm, yshift=3.65cm] {Конфигурации и Web API (XML, JSON),};
    \node [xshift=5cm, yshift=3.35cm] {вёрстка (HTML, LaTeX), аннотации};
}

\newcommand{\paradigmsFunc} {
  \paradigmsDecls

  \node (func) [process, xshift=4cm, yshift=2.5cm] {Функциональное};
}

\newcommand{\paradigmsLogical} {
  \paradigmsFunc

  \node (logical) [process, xshift=5cm, yshift=1cm] {Логическое};
}

\newcommand{\paradigmsAll} {
  \paradigmsLogical

  \node (generic) [process, xshift=-1cm, yshift=5cm] {Обобщенное};
  \node (concurrent) [process, xshift=5cm, yshift=-2cm] {Конкурентное};
  \node (module) [process, xshift=-2cm, yshift=4cm] {Модульное};
}
