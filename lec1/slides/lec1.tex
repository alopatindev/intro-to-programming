\documentclass{beamer}

\usepackage[utf8]{inputenc}
\usepackage[russian]{babel}

%\usepackage{beamerthemesplit}
\definecolor{myblue}{rgb}{.0,.2,.3}
\setbeamercolor*{palette primary}{use=structure,fg=white,bg=myblue}
\setbeamertemplate{navigation symbols}{}

\usepackage{tikz}
\usetikzlibrary{shapes.geometric, arrows}
\tikzstyle{io} = [trapezium, trapezium left angle=70, trapezium right angle=110, minimum width=1cm, minimum height=0.33cm, text centered, draw=black, fill=blue!30]
\tikzstyle{process} = [rectangle, minimum width=1cm, minimum height=0.31cm, text centered, draw=black, fill=orange!30]
\tikzstyle{decision} = [diamond, minimum width=1cm, minimum height=0.31cm, text centered, draw=black, fill=green!30]
\tikzstyle{arrow} = [thick,->,>=stealth]

\title{Введение в программирование}
\author{Лопатин Александр}
\date{2015}


\subtitle{Лекция 1}

\begin{document}

  \frame{\titlepage}


  \section*{Содержание} {
    \frame{\tableofcontents[hideallsubsections]}


  \section{О чем курс}
    \frame {
      \begin{itemize}
        \item урезанная версия одноименного универовского курса (обычно рассчитанного на 1---2 семестра)
        \item основы и упоминания из других курсов
      \end{itemize}
    }

    \frame {
      \frametitle{Expectations}
      \begin{itemize}
        \item получим кругозор
        \item узнаем основы нескольких языков
        \item научимся базовым техникам
        \item на практике применим некоторые техники
      \end{itemize}
    }


  \section{Как построен курс}
    \frame {
      \begin{itemize}
        \item разбит так, чтобы не вызвать передоз информацией
        \item одно занятие в неделю на 2---3 часа (включает лекцию и практику)
        \item 10 недель ($\approx 2\frac{1}{3}$ месяца)
        \item домашки на не более 2-х часов в неделю
        \item один мелкий проект на 2 недели (2 викенда)
        \item домашки и проект будут оцениваться
      \end{itemize}
    }


  \section{Вводные термины}
    \frame {
      \frametitle{Исполнитель}
      \begin{itemize}
        \item повар
        \item военнослужащий
        \item гитарист
        \item рок-группа
        \item коммерческая компания
        \item вычислительная система (компьютер, смартфон, бытовой прибор и т.п.)
      \end{itemize}
    }

    \frame {
      \frametitle{Инструкции исполнителя}
      \begin{itemize}
        \item \textbf{пункты рецепта} (<<42. Положить ложку сахара>>) для повара
        \item \textbf{приказы} (<<нале-во!>>) для военнослужащего
        \item \textbf{аккорды} для гитариста 
        \item \textbf{инструкции процессора} (<<сложить два числа>>) для вычислительной системы
      \end{itemize}
    }

    \frame {
      \frametitle{Алгоритм}
      Конечная последовательность \textbf{инструкций исполнителя} направленная на решение задачи
      \vspace{1cm}
      \begin{itemize}
        \item \textbf{рецепт} для повара
        \item \textbf{лабораторная работа} для студента
        \item \textbf{текст песни} для вокалиста
        \item \textbf{бизнес-план} коммерческой компании
        \item \textbf{шаги воспроизведения проблемы} для тестировщика
        \item \textbf{компьютерная программа} для вычислительной системы
      \end{itemize}
    }

    \frame {
      \frametitle{С точки зрения данных}
      Компьютерные программы представляю из себя
      \underline{ввод/вывод данных} и \underline{обработку данных}
    }

    \frame {
      \frametitle{Интерфейс}
        Означает <<взаимодействие>>

        Нужно для осуществления ввода/вывода данных
        \vspace{0.5cm}
        \begin{itemize}
          \item Пользовательский (UI --- User Interface)
            \begin{itemize}
              \item Графический пользовательский интерфейс, Graphical User Interface (GUI)
              \item Интерфейс коммандной строчки, Command-Line Interface (CLI)
            \end{itemize}
          \item Программный (API --- Application Interface)
            \begin{itemize}
              \item подпрограммы, модули, библиотеки
              \item сетевые протоколы (например HTTP)
              \item много других
            \end{itemize}
        \end{itemize}
        \vspace{0.4cm}
        CLI может использоваться как API
    }

  \section{Обзор развития вычислительных систем}
    \frame {
      \frametitle{1945: Архитектура фон Неймана (Von Neumann architecture)}
      \begin{center}
        \includegraphics[scale=1]{pictures/Von_Neumann_architecture.png}
      \end{center}
    }

    \frame {
      \frametitle{1946: ENIAC}
      \begin{center}
        \includegraphics[scale=0.4]{pictures/ENIAC.png}
      \end{center}
    }

    \frame {
      \frametitle{1964: IBM System/360}
      \begin{center}
        \includegraphics[scale=0.07]{pictures/IBM_S360.jpg}
      \end{center}
    }

%    \frame {
%      \frametitle{<<Хакерская этика>> 60-ых}
%      В 60-е значение слова <<Хакер>> еще не было испорчено журналистами
%      \begin{itemize}
%        \item Доступ к компьютерам и любым другим средствам познания устройства мира для каждого должен быть неограниченным
%        \item Информация должна быть свободной
%        \item Недоверие властям и продвижение принципа децентрализации
%        \item Оценивать хакера можно лишь по его достижениям. Ни положение в обществе, ни возраст, ни раса не играют при этом никакой роли
%      \end{itemize}
%
%      \vspace{0.5cm}
%      \includegraphics[scale=0.05]{../../common/book.png}~Стивен Леви --- Хакеры: Герои компьютерной революции
%    }

    \frame {
      \frametitle{1975: Закон Мура}
      \begin{center}
        \includegraphics[scale=0.4]{pictures/moore_law.png}
      \end{center}
    }

    \frame {
      Закон прекратил действовать --- пришло время новых идей по увеличению производительности железа

      \vspace{0.5cm}
      \begin{itemize}
        \item выполнять трудоемкие операции на других устройствах (например GPU)
        \item объединять несколько ядер процессоров в один
        \item объединять несколько компьютеров в вычислительный кластер
        \item проектировать концептуально другие выч. системы, необязательно на основе фон Неймовской архитектуры
        \begin{itemize}
          \item квантовые компьютеры
          \item клеточные компьютеры
          \item ...
        \end{itemize}
      \end{itemize}
    }

    \frame {
      С ростом производительности железа растет и сложность задач

      \vspace{0.5cm}
      Сложность программ тоже растет

      \vspace{0.5cm}
      Появляется много специализаций в IT, по аналогии с врачами
    }

    \frame {
      \frametitle{Многоуровневые системы}
      Системы состоят из подсистем (из слоёв / уровней)
      \begin{center}
        \includegraphics[scale=0.4]{../../common/you_dont_say.png}
      \end{center}
    }

    \frame {
      \layersInRealLife
    }

    \frame {
      \frametitle{Каждая железяка состоит из множества более низкоуровневых железок}
      \begin{center}
        \includegraphics[scale=0.7]{pictures/arm-arch.png}
      \end{center}
    }

    \frame {
      \frametitle{Софт тоже многоуровневый}

      \begin{center}
        \includegraphics[scale=0.7]{pictures/programs-as-seen-by-users.png}
      \end{center}
    }

  \section{Парадигмы программирования}
    \frame {
      \begin{tikzpicture}
        \paradigmsImpDecl
      \end{tikzpicture}
    }

    \frame {
      \frametitle{Императивное программирование (Imperative Programming)}

      Написание \textbf{алгоритмов} путём перечисления \textbf{инструкций исполнителя}
      \vspace{0.5cm}
      \begin{center}
        \includegraphics[scale=0.5]{pictures/imperative-programmer.jpg}
      \end{center}
    }

    \begin{frame}
      \frametitle{Декларативное программирование (Declarative Programming)}
      \textbf{Описание желаемого результата}, \underline{вместо алгоритма} получения этого результата.\linebreak
        \vspace{0.5cm}

        Например запрос к базе данных:

        \vspace{0.5cm}
        \texttt{Выбрать абитуриентов}

        \texttt{поступающих на специальность <<Программная Инженерия>>}

        \texttt{с сортировкой по сумме баллов по убыванию}
        \vspace{1cm}

        Подробнее рассмотрим ближе к концу курса
    \end{frame}


  \section{Алгоритмическое программирование}

    \frame {
      \begin{tikzpicture}
        \paradigmsAlg
      \end{tikzpicture}
    }

    \subsection{Словесное описание}
      \frame {
        \frametitle{Линейный алгоритм}
        \begin{enumerate}
          \item Шагнуть вперед
          \item Шагнуть вперед
          \item Повернуть направо на $90^\circ$
          \item Шагнуть вперед
        \end{enumerate}
      }

      \frame {
        \frametitle{Алгоритм с ветвлением}

        \vspace{0.5cm}
%        1. Для каждого студента
%        \begin{itemize}
%          \item 1.1. Напечатать имя, фамилию
%          \item 1.2. Если балл <= 2 перейти к п. 1.2.1 иначе к п. 1.3
%          \begin{itemize}
%            \item 1.2.1. Задать красный цвет шрифта
%          \end{itemize}
%          \item 1.3. Напечатать оценку
%          \item 1.4. Задать черный цвет шрифта
%        \end{itemize}
        \begin{enumerate}
          \item Если небо светлое --- перейти к п. 2 иначе к п. 3
          \item Напечатать <<Сейчас день>>
          \item Напечатать <<Сейчас ночь>>
        \end{enumerate}
      }

    \subsection{Язык блок-схем (flow-charts)}
      \frame {
        \frametitle{Линейный алгоритм}
        \begin{center}
          \begin{tikzpicture}
            \node (pro1) [process] {Шагнуть вперед};
            \node (pro2) [process, below of=pro1] {Шагнуть вперед};
            \node (pro3) [process, below of=pro2] {Повернуть направо на $90^\circ$};
            \node (pro4) [process, below of=pro3] {Шагнуть вперед};
            \draw [arrow] (pro1) -- (pro2);
            \draw [arrow] (pro2) -- (pro3);
            \draw [arrow] (pro3) -- (pro4);
          \end{tikzpicture}
        \end{center}
      }

      \newcommand{\algtree} {
        \node (dec1) [decision] {балл <= 2};
        \node [left of=dec1, node distance=2.3cm] {Да};
        \node [right of=dec1, node distance=2.3cm] {Нет};

        \node (pro1) [process, below of=dec1, left of=dec1, node distance=2cm] {Задать красный цвет};
        \draw [arrow] (dec1) -| (pro1);
      }
      \frame {
        \frametitle{Алгоритм с ветвлением}
        \framesubtitle{Бывает с одной веткой}
        \begin{center}
          \begin{tikzpicture}
            \algtree
            \node (pro2) [below of=dec1, right of=dec1, node distance=2cm] {};
            \node (pro3) [process, below of=pro1, right of=pro1, node distance=2cm] {Напечатать оценку};
            \draw [arrow] (dec1) -| (pro2);
            \draw [arrow] (pro1) |- (pro3);
            \draw [arrow] (pro2) |- (pro3);
          \end{tikzpicture}
        \end{center}
      }

      \frame {
        \frametitle{Алгоритм с ветвлением}
        \framesubtitle{И с двумя ветками}
        \begin{center}
          \begin{tikzpicture}
            \algtree
            \node (pro2) [process, below of=dec1, right of=dec1, node distance=2cm] {Задать черный цвет};
            \node (pro3) [process, below of=pro1, right of=pro1, node distance=2cm] {Напечатать оценку};

            \draw [arrow] (dec1) -| (pro2);
            \draw [arrow] (pro1) |- (pro3);
            \draw [arrow] (pro2) |- (pro3);
          \end{tikzpicture}
        \end{center}
      }

      \frame {
        \frametitle{Практика}
        \begin{center}
          \begin{tikzpicture}
            \node (io1) [io] {a, b};
            \node (threedots) [below of=io1] {...};
            \node (io2) [io, below of=threedots] {c};
            \draw [arrow] (io1) -- (threedots);
            \draw [arrow] (threedots) -- (io2);
          \end{tikzpicture}
        \end{center}

        \begin{enumerate}
          \item Нарисовать блок-схему к программе, которая берет в качестве ввода числа $a$ и $b$ и решает уравнение $a = b \cdot c$. Вывести решение (значение $c$). (Подсказка: сначала нужно вручную преобразовать уравнение)
          \item Предусмотреть обработку ошибки деления на ноль к предыдущему заданию. Выводить <<Ошибка: деление на ноль>> вместо решения в этом случае
          \item Написать словесное описание получившегося алгоритма
        \end{enumerate}
      }


  \section{Структурное программирование}
    \frame {
      \begin{tikzpicture}
        \paradigmsStruct
      \end{tikzpicture}
    }

    \frame {
      \frametitle{Структурное программирование (Structured Programming)}
      \begin{itemize}
        \item отказ от меток и безусловного перехода
        \item использование двух \textbf{управляющих структур}
        \begin{itemize}
          \item условие
          \item цикл
        \end{itemize}
      \end{itemize}
    }

    \frame {
      \begin{center}
        \includegraphics[scale=0.4]{pictures/spaghetti-example.png}
      \end{center}
    }

    \frame {
      \frametitle{Spaghetti is write-only code}
      \begin{center}
        \includegraphics[scale=0.3]{pictures/spaghetti-flowchart.png}
      \end{center}
    }

    \frame {
      \frametitle{Повторения всё же нужны}
      \framesubtitle{Но не такие безобразные}
      \begin{center}
        \begin{figure}
          \includegraphics[scale=0.5]{pictures/spaghetti-red.png}
          \includegraphics[scale=0.5]{pictures/structured-red.png}
          \caption{Спагетти-код и структурный код}
        \end{figure}
      \end{center}
    }

    \newcommand{\loopTypes} {
      \begin{center}
        \includegraphics[scale=2]{pictures/loops.png}
      \end{center}
    }

    \frame {
      \frametitle{Циклы}
      \loopTypes
    }

    \frame[label=preConditionLoop] {
      \frametitle{Цикл с предусловием}
      \begin{enumerate}
        \item Пока гвоздь не забит
        \begin{enumerate}
          \item Поднять молоток
          \item Ударить молотком по гвоздю
          \item Перейти к п. 1
        \end{enumerate}
      \end{enumerate}
      \loopTypes
    }

    \frame[label=postConditionLoop] {
      \frametitle{Цикл с постусловием}
      \begin{enumerate}
        \item (Делать)
        \begin{enumerate}
          \item Поднять молоток
          \item Ударить молотком по гвоздю
        \end{enumerate}
        \item Пока гвоздь не забит --- перейти к п. 1
      \end{enumerate}
      \loopTypes
    }

    \frame {
      \frametitle{Цикл со счетчиком --- частный случай цикла с предусловием}
      \begin{enumerate}
        \item Инициализировать счетчик i значением 0
        \item Пока i < n
        \begin{enumerate}
          \item Выполнить действие
          \item Выполнить другое действие
          \item ...
          \item Увеличить счетчик i на единицу
          \item Перейти к п. 2.
        \end{enumerate}
      \end{enumerate}
    }

    \frame[label=innerLoops] {
      \frametitle{Циклы могут быть вложены}
      \begin{enumerate}
        \item i = 1
        \item Пока i <= количество\_студентов
        \item Напечатать имя i-го студента
        \begin{enumerate}
          \item j = 1
          \item Пока j <= количество\_оценок\_студента
          \begin{enumerate}
            \item Показать j-ю оценку i-го студента
            \item Перейти к п. 3.2
          \end{enumerate}
          \item Перейти к п. 2
        \end{enumerate}
        \item Напечатать <<такие дела>> 
      \end{enumerate}
    }

    \frame {
      \frametitle{Практика}
      \begin{enumerate}
        \item Нарисовать блок-схему к алгоритму забивания гвоздя \underline{\hyperlink{preConditionLoop}{со слайда с циклом с предусловием}}
        \item Тоже самое для \underline{\hyperlink{postConditionLoop}{слайда с циклом с постусловием}}
        \item Тоже самое для \underline{\hyperlink{innerLoops}{алгоритма с вложенными циклами}} (подсказка: можно начать с вложенного цикла)
      \end{enumerate}
    }

    \frame {
      \frametitle{Домашка}
      Нарисовать \underline{блок-схему} и написать \underline{словесное описание} к алгоритму решения квадратного уравнения $ax^2 + bx + c = 0$.

      Входные данные: $a$, $b$ и $c$. Выходные данные:
      \begin{itemize}
        \item либо <<Корней нет>>
        \item либо $x_1$
        \item либо $x_1$ и $x_2$
      \end{itemize}

      Подсказки:
      \begin{itemize}
        \item дискриминант вычисляется по формуле $D = b^2 - 4\cdot a \cdot c$
        \item корни уравнения вычисляются по формуле $x_{1,2} = \frac{-b \pm \sqrt{D}}{2 \cdot a}$
        \item количество корней зависит от значений дискриминанта (3~ситуации:~$D < 0$, $D = 0$ и $D > 0$)
      \end{itemize}
    }


\end{document}

% vim:shiftwidth=2:softtabstop=2:tabstop=2
