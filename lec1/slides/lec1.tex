\documentclass{beamer}

\usepackage[utf8]{inputenc}
\usepackage[russian]{babel}
%\usepackage{beamerthemesplit}

\definecolor{myblue}{rgb}{.0,.2,.3}
\setbeamercolor*{palette primary}{use=structure,fg=white,bg=myblue}

\setbeamertemplate{navigation symbols}{}

\title{Введение в программирование}
\author{Лопатин Александр}
\date{2015}

\begin{document}
  \frame{\titlepage}


  \section*{Содержание}
  \frame{\tableofcontents}


  \section{Парадигмы программирования}

    \frame {
      \frametitle{Императивное программирование}
    }

    \frame {
      \frametitle{Декларативное программирование}
    }

    \subsection{Императивное программирование}

      \frame {
        \frametitle{Алгоритмическое программирование}
      }

      \frame {
        \frametitle{Структурное программирование}
      }

      \frame {
        \frametitle{Процедурное программирование}
      }

      \frame {
        \frametitle{Объектно-ориентированное программирование (ООП)}
      }

    \subsection{Декларативное программирование}
      \frame {
      }

\end{document}

% vim:shiftwidth=2:softtabstop=2:tabstop=2
