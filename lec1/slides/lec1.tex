\documentclass{beamer}

\usepackage[utf8]{inputenc}
\usepackage[russian]{babel}

%\usepackage{beamerthemesplit}
\definecolor{myblue}{rgb}{.0,.2,.3}
\setbeamercolor*{palette primary}{use=structure,fg=white,bg=myblue}
\setbeamertemplate{navigation symbols}{}

\usepackage{tikz}
\usetikzlibrary{shapes.geometric, arrows}
\tikzstyle{io} = [trapezium, trapezium left angle=70, trapezium right angle=110, minimum width=1cm, minimum height=0.33cm, text centered, draw=black, fill=blue!30]
\tikzstyle{process} = [rectangle, minimum width=1cm, minimum height=0.31cm, text centered, draw=black, fill=orange!30]
\tikzstyle{decision} = [diamond, minimum width=1cm, minimum height=0.31cm, text centered, draw=black, fill=green!30]
\tikzstyle{arrow} = [thick,->,>=stealth]

\title{Введение в программирование}
\author{Лопатин Александр}
\date{2015}


\begin{document}

  \frame{\titlepage}


  \section*{Содержание}
    \frame{\tableofcontents[hideallsubsections]}


  \section{Вводные термины}
    \frame {
      \frametitle{Исполнитель}
      \begin{itemize}
        \item повар
        \item военнослужащий
        \item гитарист
        \item рок-группа
        \item коммерческая компания
        \item вычислительная система (компьютер, смартфон, бытовой прибор и т.п.)
      \end{itemize}
    }

    \frame {
      \frametitle{Инструкции исполнителя}
      \begin{itemize}
        \item \textbf{пункты рецепта} (<<42. Положить ложку сахара>>) для повара
        \item \textbf{приказы} (<<нале-во!>>) для военнослужащего
        \item \textbf{аккорды} для гитариста 
        \item \textbf{инструкции процессора} (<<сложить два числа>>) для вычислительной системы
      \end{itemize}
    }

    \frame {
      \frametitle{Алгоритм}
      Конечная последовательность \textbf{инструкций исполнителя} направленная на решение задачи
      \vspace{1cm}
      \begin{itemize}
        \item \textbf{рецепт} для повара
        \item \textbf{лабораторная работа} для студента
        \item \textbf{текст песни} для вокалиста
        \item \textbf{бизнес-план} коммерческой компании
        \item \textbf{шаги воспроизведения проблемы} для тестировщика
        \item \textbf{компьютерная программа} для вычислительной системы
      \end{itemize}
    }

    \frame {
      \frametitle{Интерфейс}
        %Взаимодействие
        \begin{itemize}
          \item Пользовательский (UI --- User Interface)
            %\subitem Graphical User Interface GUI
            %\subitem Command-Line Interface (CLI)
          \item Программный (API --- Application Interface)
        \end{itemize}
        \vspace{1cm}
        CLI может использоваться как API
    }

  \section{Обзор развития вычислительных систем}
    \frame {
    }
    \frame {
      \frametitle{Многоуровневые системы}
    }


  \section{Парадигмы программирования}
    \frame {
      \begin{tikzpicture}
        \node (imper) [bigrect, xshift=-1cm] {Императивное};
          \node (alg) [process, dashed, xshift=-1cm, yshift=1cm] {Алгоритмическое};
            \node [xshift=-1cm, yshift=0.65cm] {Блок-схемы, словесное описание};
          \node (struct) [process, xshift=0cm, yshift=0cm] {Структурное};
            \node [xshift=0cm, yshift=-0.35cm] {Fortran...};
          \node (proc) [process, xshift=-1cm, yshift=-1cm] {Процедурное};
            \node [xshift=-1cm, yshift=-1.35cm] {C, Pascal...};
          \node (oop) [process, xshift=-1.5cm, yshift=-2cm] {Объектно-ориентированное (ООП)};
            \node [xshift=-1.5cm, yshift=-2.35cm] {C++, C\#, Java, Python, Ruby, JS, ...};

        \node (decl) [bigrect, xshift=5cm] {Декларативное};

        \draw [arrow] (alg) -- (struct);
        \draw [arrow] (struct) -- (proc);
        \draw [arrow] (proc) -- (oop);
      \end{tikzpicture}
    }

    \subsection{Императивное программирование (Imperative Programming)}
    \begin{frame}
      Написание \textbf{алгоритмов} перечислением \textbf{инструкций исполнителя}
        \vspace{1cm}
        \texttt{1. Шагнуть вперед}

        \texttt{2. Повернуть направо}

        \texttt{3. Шагнуть вперед}
        \vspace{1cm}
    \end{frame}

    \subsection{Декларативное программирование (Declarative Programming)}
    \begin{frame}
      \textbf{Описание желаемого результата}, \underline{вместо алгоритма} получения этого результата.\linebreak
        \vspace{1cm}

        Например запрос к базе данных:

        \vspace{0.5cm}
        \texttt{Выбрать абитуриентов}

        \texttt{поступающих на специальность Программная Инженерия}

        \texttt{с сортировкой по сумме баллов}
        \vspace{1cm}

        Подробнее рассмотрим ближе к концу курса
    \end{frame}


  \section{Алгоритмическое программирование}

    \subsection{Словесное описание}
      \frame {
        \frametitle{Линейный алгоритм}
      }

      \frame {
        \frametitle{Алгоритм с ветвлением}
      }

    \subsection{Язык блок-схем}
      \frame {
        \frametitle{Линейный алгоритм}
      }

      \frame {
        \frametitle{Алгоритм с ветвлением}
      }

%      \frame {
%        \begin{tikzpicture}
%          \node (in1) [io] {Input};
%          \node (pro1) [process, below of=in1] {Process 1};
%          \node (dec1) [decision, below of=pro1, yshift=-0.5cm] {Decision 1};
%          \node (pro2a) [process, below of=dec1, yshift=-0.5cm] {Process 2a};
%          \node (pro2b) [process, right of=dec1, xshift=2cm] {Process 2b};
%          \node (out1) [io, below of=pro2a] {Output};
%          \draw [arrow] (pro2b) |- (pro1);
%          \draw [arrow] (pro2a) -- (out1);
%        \end{tikzpicture}
%      }

      \frame {
        \frametitle{Практика}
      }


  \section{Структурное программирование}
    \frame {
      \frametitle{Structured Programming}
      \begin{itemize}
        \item отказ от безусловного перехода
        \item использование двух \textbf{управляющих структур}
          %\subitem условие
          %\subitem цикл
      \end{itemize}
    }

    \frame {
      \frametitle{Пример алгоритма}
    }

    \frame {
      \frametitle{Практика}
    }


\end{document}

% vim:shiftwidth=2:softtabstop=2:tabstop=2
