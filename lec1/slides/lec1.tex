\documentclass{beamer}

\usepackage[utf8]{inputenc}
\usepackage[russian]{babel}

%\usepackage{beamerthemesplit}
\definecolor{myblue}{rgb}{.0,.2,.3}
\setbeamercolor*{palette primary}{use=structure,fg=white,bg=myblue}
\setbeamertemplate{navigation symbols}{}

\usepackage{tikz}
\usetikzlibrary{shapes.geometric, arrows}
\tikzstyle{io} = [trapezium, trapezium left angle=70, trapezium right angle=110, minimum width=1cm, minimum height=0.33cm, text centered, draw=black, fill=blue!30]
\tikzstyle{process} = [rectangle, minimum width=1cm, minimum height=0.31cm, text centered, draw=black, fill=orange!30]
\tikzstyle{decision} = [diamond, minimum width=1cm, minimum height=0.31cm, text centered, draw=black, fill=green!30]
\tikzstyle{arrow} = [thick,->,>=stealth]

\title{Введение в программирование}
\author{Лопатин Александр}
\date{2015}


\begin{document}
  \frame{\titlepage}


  \section*{Содержание}
  \frame{\tableofcontents}


  \section{Парадигмы программирования}

    \frame {
      \frametitle{Императивное программирование}
    }

    \frame {
      \frametitle{Декларативное программирование}
    }

    \subsection{Императивное программирование}

      \frame {
        \frametitle{Алгоритмическое программирование}

        \begin{tikzpicture}[node distance=1.3cm]
          \node (in1) [io] {Input};
          \node (pro1) [process, below of=in1] {Process 1};
          \node (dec1) [decision, below of=pro1, yshift=-0.5cm] {Decision 1};
          \node (pro2a) [process, below of=dec1, yshift=-0.5cm] {Process 2a};
          \node (pro2b) [process, right of=dec1, xshift=2cm] {Process 2b};
          \node (out1) [io, below of=pro2a] {Output};
          \draw [arrow] (pro2b) -- (pro1);
          \draw [arrow] (pro2a) -- (out1);
        \end{tikzpicture}

      }

      \frame {
        \frametitle{Структурное программирование}
      }

      \frame {
        \frametitle{Процедурное программирование}
      }

      \frame {
        \frametitle{Объектно-ориентированное программирование (ООП)}
      }

    \subsection{Декларативное программирование}
      \frame {
      }

\end{document}

% vim:shiftwidth=2:softtabstop=2:tabstop=2
