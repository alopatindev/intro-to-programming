\documentclass{beamer}

\usepackage[utf8]{inputenc}
\usepackage[russian]{babel}

%\usepackage{beamerthemesplit}
\definecolor{myblue}{rgb}{.0,.2,.3}
\setbeamercolor*{palette primary}{use=structure,fg=white,bg=myblue}
\setbeamertemplate{navigation symbols}{}

\usepackage{tikz}
\usetikzlibrary{shapes.geometric, arrows}
\tikzstyle{io} = [trapezium, trapezium left angle=70, trapezium right angle=110, minimum width=1cm, minimum height=0.33cm, text centered, draw=black, fill=blue!30]
\tikzstyle{process} = [rectangle, minimum width=1cm, minimum height=0.31cm, text centered, draw=black, fill=orange!30]
\tikzstyle{decision} = [diamond, minimum width=1cm, minimum height=0.31cm, text centered, draw=black, fill=green!30]
\tikzstyle{arrow} = [thick,->,>=stealth]

\title{Введение в программирование}
\author{Лопатин Александр}
\date{2015}


\subtitle{Лекция 2}

\begin{document}

  \frame{\titlepage}


  \section*{Содержание} {
    \frame{\tableofcontents[hideallsubsections]}


  \section{Языки программирования}
    \frame {
      \frametitle{Язык программирования}
      Знаковая система \underline{для написания компьютерных программ}

      \vspace{0.5cm}
      Текст, написанный на таком языке называют \underline{текст программы} или \underline{исходный~код}~(source~code) или просто \underline{код}
    }

    \frame {
      \frametitle{Трансляторы языков программирования}
      Программы, которые \underline{понимают исходный код}

      \vspace{0.5cm}
      Бывают двух типов:
      \begin{itemize}
        \item Компилятор --- превращает текст программы в \underline{двоичный код}
        \item Интерпретатор --- читает текст программы и выполняет написанное
      \end{itemize}
    }

    \frame {
      \frametitle{Двоичный код}
      \begin{itemize}
        \item \textbf{машинный код} (machine/native code) --- код, который исполняет \underline{аппаратный} исполнитель (например микропроцессор)
        \item \textbf{байт-код} \underline{виртуальной машины} --- код, который исполняет \underline{программный} исполнитель (например <<виртуальная машина .NET>> или <<виртуальная машина Java (JVM)>>) 
      \end{itemize}

      \vspace{0.5cm}
      Понятие <<виртуальная машина>> --- многозначно.
      Об этом разжевано здесь:
      \url{http://habrahabr.ru/company/intel/blog/254793/}
    }

    \frame {
      С ростом популярности \underline{JIT-компиляции} разделение \underline{трансляторов} на
      \underline{компиляторы} и \underline{интерпретаторы} не столь актуально

      \vspace{1cm}
      Большинство \underline{актуальных интерпретаторов} стало, грубо говоря,
      \underline{компиляторами} в \underline{машинный код})

      \vspace{1cm}
      \url{https://ru.wikipedia.org/wiki/JIT}
    }

    \frame {
      Лучше разделять сами языки, а не трансляторы:
      \begin{itemize}
        \item \textbf{Компилируемые} (Assembler, C++, C\#, Java, ...)
        \item \textbf{Скриптовые} (Python, Ruby, JavaScript/ECMAScript, bash/shell, cmd/bat, PowerShell, ...)
      \end{itemize}
    }

    \frame {
      \frametitle{Компилируемые языки}
      Исходный код компилируют в \underline{двоичный код} и (обычно) распространяют в скомпилированном виде (.exe, .jar, .dll и т.д.)

      \vspace{0.5cm}
      Часто используется для:
      \begin{itemize}
        \item \textbf{прикладного программирования} (от текстового редактора до веб-браузера или более сложной системы)
        \item \textbf{системного программирования} (драйвера устройств и т.д.)
      \end{itemize}
    }

    \frame {
      \frametitle{Скриптовые языки (языки сценариев)}
      \framesubtitle{Для многих --- синоним к <<Интерпретируемым языкам>>}
      Программы обычно распространяют в виде исходного кода (.py,~.js,~.bat~и~т.д.)

      \vspace{0.5cm}
      Часто используется для:
      \begin{itemize}
        \item \textbf{прототипов} прикладных программ
        \item \textbf{сценариев} для автоматизации: сборка/тестирование билда, deploy на сервер
        \item \textbf{плагинов/расширений} (для браузера, скажем)
        \item клиентский или серверный \textbf{код для веб-сайта}
      \end{itemize}

      \vspace{0.5cm}
      Популярно там, где хочется \underline{быстро увидеть результат}
    }

    \frame {
      А еще языки классифицируют по тому, на какие \textbf{парадигмы~программирования} сделан основной акцент...
    }

    \frame {
      \frametitle{Wait, wait...}
      \framesubtitle{Зачем об этом всём знать? Для осознания того}

      \begin{itemize}
        \item что языки можно \underline{очень по-разному} классифицировать
        \item \underline{одни классы} эффективно решают \underline{одни типы задач}, другие классы --- другие типы задач
        \item{} <<универсального>> языка \underline{не существует}
        \item чем более \underline{узкоспециализирован} язык (SQL для БД, G-Codes для станков, ...) или комбинация языка и библиотеки/фреймворка (Ruby+RoR или JS+node.js для веб, ...) тем \underline{быстрее} их можно изучить
      \end{itemize}

      Другими словами --- чтобы знать по какому принципу выбирать следующий язык для изучения

      \vspace{0.5cm}
      \url{https://www.youtube.com/watch?v=LR8fQiskYII}
      \url{https://www.youtube.com/watch?v=NvWTnIoQZj4}
    }

    \frame {
      \frametitle{На википедии}
      Можно увидеть разницу между языками, если обращать внимание на \underline{Paradigm} и \underline{Typing discipline}
      \begin{center}
        \includegraphics[scale=0.18]{pictures/python.png}
        \includegraphics[scale=0.2]{pictures/js.png}
        \includegraphics[scale=0.12]{pictures/java.png}
      \end{center}
    }

  \section{<<Hello world>> на всех языках}
\begin{frame}[fragile]
  \frametitle{Привет, Python 2!}
  \begin{minted}{python}
print "Hello World!"
  \end{minted}

  \vspace{2cm}
  Попробовать --- \url{http://tutorialspoint.com/execute\_python\_online.php}
\end{frame}

\begin{frame}[fragile]
  \frametitle{Привет, JavaScript (ECMAScript 6)!}
  \begin{minted}{js}
console.log("Hello World!")
  \end{minted}

  \vspace{2cm}
  Попробовать --- \url{http://www.es6fiddle.net/}
\end{frame}

\begin{frame}[fragile]
  \frametitle{Привет, Java 8!}
  \begin{minted}{java}
public class HelloWorld {
    public static void main(String[] args) {
        System.out.println("Hello World!");
    }
}
  \end{minted}

  \vspace{2cm}
  Попробовать --- %\url{http://www.tutorialspoint.com/compile\_java8\_online.php}
\url{http://www.compilejava.net/}
\end{frame}

\section{Переменные и константы}
\begin{frame}[fragile]
  \frametitle{Данные в языках --- это переменные и константы}
  \textbf{Переменную} можно изменять (время, координаты, ...)

  Пример на Java:
  \begin{minted}{java}
int x = 2;
System.out.println("x == " + x);       // x == 2
x = x + 1;
System.out.println("x == " + x);       // x == 3
  \end{minted}
  \vspace{1cm}
  \textbf{Константу} нельзя изменять (число $\pi$, скорость света ...)
  Пример на Java:
  \begin{minted}{java}
final double pi = 3.14;
  \end{minted}
\end{frame}

\frame {
  \frametitle{Данные могут быть разных типов}
  Примитивные типы:
  \begin{itemize}
    \item \textbf{целое} число (\underline{int}, long, ...): -123, 12345678910L
    \item \textbf{дробное} число (\underline{float}, double): 0.123f, 0.123
    \item \textbf{символ} (char): 'z'
  \end{itemize}

  \vspace{1cm}
  Составные типы:
  \begin{itemize}
    \item \textbf{массивы} (\underline{array}): [1, 2, 3, 4]
    \item \textbf{объекты} (\underline{string}, list, set, dict/map, ...): $"$Hello$"$, \{"hello":~"привет"\}
  \end{itemize}
}

\section{Действия (операторы)}
\begin{frame}[fragile]
  \frametitle{Присваивание (Python)}
  \begin{minted}{python}
x = 1
y = 2
print x       # 1
x = y
print x       # 2
  \end{minted}

  \vspace{0.5cm}
Синтаксис: переменная = выражение

Читается <<переменной присвоить выражение>>
\end{frame}

\begin{frame}[fragile]
  \frametitle{Арифметические операторы (Python)}
  \begin{minted}{python}
x = 2 + 2
print x               # 4
x = x - 1
print x               # 3
x = x * x
print x               # 9
print x % 2           # 1 (остаток от деления)
print x / 2           # 4 (целочисленное деление)
print x / 2.0         # 4.5 (деление целого на дробное)
print float(x) / 2.0  # 4.5
  \end{minted}
В предпоследнем: целое (int) было неявно преобразовано в дробное (float)
\end{frame}

\begin{frame}[fragile]
  \frametitle{Логические операторы (Python)}
  \begin{minted}{python}
x = True
y = False
print x and y         # False
print x or y          # True
print not x           # False (тоже самое, что и x == False)

a = 2
b = 3
print a < b           # True
print a > b           # False
print a <= b, a >= b  # True False
print a == b          # False (читается "a равняется b")
print a == (b - 1)    # True
  \end{minted}
  Не путать \textbf{равенство} (логический оператор) с \textbf{присвоением} (действие, которое изменяет значение переменной)
\end{frame}

\begin{frame}[fragile]
  \frametitle{Логические операторы (JavaScript)}
  \begin{minted}{js}
var x = true
var y = false
console.log(x && y)    // false
console.log(x || y)    // true
console.log(!x)        // false

var a = 2
var b = 3
// операторы сравнения везде одинаковые, кроме равенства
console.log(a === b)   // false
  \end{minted}
\end{frame}

\begin{frame}[fragile]
  \frametitle{Ввод (Python)}
  \begin{minted}{python}
line = raw_input()         # ввод строки
number = int(line)         # преобразование в целое число
line = raw_input()         # снова ввод строки
floatNumber = float(line)  # преобразование в дробное число
  \end{minted}
\end{frame}

\begin{frame}[fragile]
  \frametitle{Массивы (Python)}
  На самом деле это нечто больше чем массивы, их используют как списки (List), стеки (Stack)
  и очереди (Queue), на которые больше внимания будет уделено в Лекции 4
  \vspace{0.5cm}
  \begin{minted}{python}
x = [1, 2, 55, -123]
print x[0]            # 1
print x[2]            # 55
x[2] = 777
print x               # [1, 2, 777, -55]
print len(x)          # 4
  \end{minted}

  \vspace{0.5cm}
  Подробней --- \url{https://docs.python.org/2/tutorial/datastructures.html}
\end{frame}

\begin{frame}[fragile]
  \frametitle{Массивы (Java)}
  \begin{minted}{java}
int x[] = {1, 2, 55, -123};
int y[] = new int[x.length];
System.out.println("length is " + x.length);  // length is 4
  \end{minted}

  \vspace{0.5cm}
  Подробней --- \url{http://www.tutorialspoint.com/java/java\_arrays.htm}
\end{frame}

\begin{frame}[fragile]
  \frametitle{Массивы (JavaScript)}
  \begin{minted}{js}
var x = [1, 2, 55, -123]
// индексировать, изменять и получать длину (aka размер) --- как в Java
  \end{minted}

  \vspace{0.5cm}
  Подробней --- \url{http://www.w3schools.com/js/js\_arrays.asp}
\end{frame}

\begin{frame}[fragile]
  \frametitle{Массивы могут быть вложены (Python)}
  Массивы с двумя уровнями вложенности называют \underline{двумерными} или
  <<массив \underline{размерности} два>>

  (не путать с \underline{размерность} с \underline{размером})

  \vspace{0.5cm}
  \begin{minted}{python}
x = [[1, 2, 55, -123], [4, 5, 6, 7]]
x[1][3] = 4444
print x
[[1, 2, 55, -123], [4, 5, 6, 4444]]
  \end{minted}
\end{frame}

\begin{frame}[fragile]
  \frametitle{Ввод (Java)}
  \begin{minted}[fontsize=\tiny]{java}
import java.util.Scanner;                      // импорт библиотеки
...
Scanner in = new Scanner(System.in);           // создание объекта
String line = in.nextLine();                   // получение строчки текста
int number = in.nextInt();                     // получение целого числа
float floatNumber = in.nextFloat();            // получение дробного числа
  \end{minted}
\end{frame}

\frame {
  \frametitle{Ввод (JavaScript)}
  Там для этого можно использовать HTML-форму

  \vspace{0.5cm}
  Пока не будем это использовать
}

\frame {
  \frametitle{Практика}

  Поиграться с описанным выше на всех языках
}

\section{Структурное программирование}
\begin{frame}[fragile]
\frametitle{Условия (JavaScript)}
  \begin{minted}{js}
var a = true
var b = false

// с одной веткой
if (a && b) {
  console.log("both are true")
}

// с двумя ветками
if (a && b) {
  console.log("both are true")
} else {
  console.log("one of them is false")
}
  \end{minted}
\end{frame}

\begin{frame}[fragile]
\frametitle{Вложенные условия (JavaScript)}
  \begin{minted}[fontsize=\tiny]{js}
var a = true
var b = false

if (a && b) {
  console.log("both are true")
} else {
  if (!a) {
    console.log("a is false")
  } else {
    console.log("b is false")
  }
}

// более читаемый вариант
if (a && b) {
  console.log("both are true")
} else if (!a) {
  console.log("a is false")
} else {
  console.log("b is false")
}
  \end{minted}
  Hint: надо всегда выделять ветки условий в фигурные скобки в языках JS и Java
\end{frame}

\begin{frame}[fragile]
\frametitle{Вложенные условия (Python)}
  \begin{minted}{python}
a = True
b = False

if a and b:
  print "both are true"
elif not a:
  print "a is false"
else:
  print "b is false"
  \end{minted}
\end{frame}

\begin{frame}[fragile]
\frametitle{Цикл с предусловием (Python)}
  \begin{minted}{python}
i = 0
while i < 10:
  print i
  i = i + 1
  \end{minted}
\end{frame}

\begin{frame}[fragile]
  \frametitle{Цикл с постусловием (Java)}
  \begin{minted}{java}
int i = 0;
do {
  System.out.println("i = " + i)
  i = i + 1;
} while (i < 10);
  \end{minted}
\end{frame}

\begin{frame}[fragile]
  \frametitle{Цикл со счетчиком (Java)}
  \begin{minted}{java}
for (int i = 0; i < 10; i = i + 1) {
  System.out.println("i = " + i)
}
  \end{minted}
\end{frame}

\begin{frame}[fragile]
  \frametitle{Цикл со <<счетчиком>> (Python)}
  \begin{minted}{python}
for i in range(0, 10, 1):
  print "i = " + str(i)
  \end{minted}
\end{frame}

\begin{frame}[fragile]
  \frametitle{Циклы могут быть вложены (Python)}
  \begin{minted}{python}
a = [[4, 5, 6], [7, 8, 9]]
for i in range(0, 2, 1):
  j = 0
  while j < 3:
    print "i =", i, " j =", j
    print a[i][j]
    j = j + 1
  \end{minted}
\end{frame}

    \frame {
      \frametitle{Домашка}

        \begin{enumerate}
          \item Реализовать задачу из предыдущей лекции (решение квадратного уравнения) на
            \underline{любом} из трёх языков программирования: Python, JavaScript или Java
            \begin{itemize}
              \item по +1 баллу за реализацию на других языках (из этих трёх)
            \end{itemize}
          \item Пользователь вводит 8 дробных чисел: 4 для матрицы A и 4 для матрицы B. Найти произведение этих матриц
            \begin{itemize}
              \item матрицы закодировать как двумерные массивы
              \item произведение матриц рассчитывается по формуле TODO
            \end{itemize}
        \end{enumerate}
    }
\end{document}
