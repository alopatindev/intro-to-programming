\documentclass{beamer}

\usepackage[utf8]{inputenc}
\usepackage[russian]{babel}

%\usepackage{beamerthemesplit}
\definecolor{myblue}{rgb}{.0,.2,.3}
\setbeamercolor*{palette primary}{use=structure,fg=white,bg=myblue}
\setbeamertemplate{navigation symbols}{}

\usepackage{tikz}
\usetikzlibrary{shapes.geometric, arrows}
\tikzstyle{io} = [trapezium, trapezium left angle=70, trapezium right angle=110, minimum width=1cm, minimum height=0.33cm, text centered, draw=black, fill=blue!30]
\tikzstyle{process} = [rectangle, minimum width=1cm, minimum height=0.31cm, text centered, draw=black, fill=orange!30]
\tikzstyle{decision} = [diamond, minimum width=1cm, minimum height=0.31cm, text centered, draw=black, fill=green!30]
\tikzstyle{arrow} = [thick,->,>=stealth]

\title{Введение в программирование}
\author{Лопатин Александр}
\date{2015}


\subtitle{Лекция 2}

\begin{document}

  \frame{\titlepage}


  \section*{Содержание} {
    \frame{\tableofcontents[hideallsubsections]}


  \section{Языки программирования}

    \frame {
      \frametitle{Трансляторы языков программирования}
      \begin{itemize}
        \item Компилятор --- превращает текст программы в \underline{двоичный код}
        \item Интерпретатор --- читает текст программы и выполняет написанное
      \end{itemize}
      \vspace{0.5cm}
      \underline{Текст программы} --- тоже самое что и \underline{исходный~код}~(source~code) или просто \underline{код}
    }

    \frame {
      С появлением JIT-компиляции такое разделение перестало быть актуальным

      \vspace{0.5cm}
      Лучше разделять сами языки, а не трансляторы:
      \begin{itemize}
        \item \textbf{Компилируемые} (Assembler, C++, C\#, Java, ...) --- код компилируют в \underline{двоичный код} и (обычно) распространяют в скомпилированном виде (.exe, .jar, .dll и т.д.)
        \item \textbf{Скриптовые} (Python, Ruby, JavaScript/ECMAScript, bash/shell, cmd/bat, PowerShell, ...) --- программы обычно распространяют в виде исходного кода (.py, .js, .bat и т.д.)
      \end{itemize}

      \vspace{0.5cm}
      \url{https://ru.wikipedia.org/wiki/JIT}
    }

    \frame {
      \frametitle{Компилируемые}
      Компилируют в двоичный код:
      \begin{itemize}
        \item \textbf{машинный код} (machine/native code) --- код, который исполняет \underline{аппаратный} исполнитель (например микропроцессор)
        \item \textbf{байт-код} \underline{виртуальной машины} --- код, который исполняет \underline{программный} исполнитель (например <<виртуальная машина .NET>> или <<виртуальная машина Java (JVM)>>) 
      \end{itemize}

      \vspace{0.5cm}
      Понятие <<виртуальная машина>> --- многозначно.
      Об этом разжевано здесь:
      \url{http://habrahabr.ru/company/intel/blog/254793/}
    }
\end{document}
