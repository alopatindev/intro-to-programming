\documentclass{beamer}

\usepackage[utf8]{inputenc}
\usepackage[russian]{babel}

%\usepackage{beamerthemesplit}
\definecolor{myblue}{rgb}{.0,.2,.3}
\setbeamercolor*{palette primary}{use=structure,fg=white,bg=myblue}
\setbeamertemplate{navigation symbols}{}

\usepackage{tikz}
\usetikzlibrary{shapes.geometric, arrows}
\tikzstyle{io} = [trapezium, trapezium left angle=70, trapezium right angle=110, minimum width=1cm, minimum height=0.33cm, text centered, draw=black, fill=blue!30]
\tikzstyle{process} = [rectangle, minimum width=1cm, minimum height=0.31cm, text centered, draw=black, fill=orange!30]
\tikzstyle{decision} = [diamond, minimum width=1cm, minimum height=0.31cm, text centered, draw=black, fill=green!30]
\tikzstyle{arrow} = [thick,->,>=stealth]

\title{Введение в программирование}
\author{Лопатин Александр}
\date{2015}


\subtitle{Лекция 2}

\begin{document}

  \frame{\titlepage}


  \section*{Содержание} {
    \frame{\tableofcontents[hideallsubsections]}


  \section{Языки программирования}
    \frame {
      \frametitle{Язык программирования}
      Знаковая система \underline{для написания компьютерных программ}

      \vspace{0.5cm}
      Текст, написанный на таком языке называют \underline{текст программы} или \underline{исходный~код}~(source~code) или просто \underline{код}
    }

    \frame {
      \frametitle{Трансляторы языков программирования}
      Программы, которые \underline{понимают исходный код}

      \vspace{0.5cm}
      Бывают двух типов:
      \begin{itemize}
        \item Компилятор --- превращает текст программы в \underline{двоичный код}
        \item Интерпретатор --- читает текст программы и выполняет написанное
      \end{itemize}
    }

    \frame {
      \frametitle{Двоичный код}
      \begin{itemize}
        \item \textbf{машинный код} (machine/native code) --- код, который исполняет \underline{аппаратный} исполнитель (например микропроцессор)
        \item \textbf{байт-код} \underline{виртуальной машины} --- код, который исполняет \underline{программный} исполнитель (например <<виртуальная машина .NET>> или <<виртуальная машина Java (JVM)>>) 
      \end{itemize}

      \vspace{0.5cm}
      Понятие <<виртуальная машина>> --- многозначно.
      Об этом разжевано здесь:
      \url{http://habrahabr.ru/company/intel/blog/254793/}
    }

    \frame {
      С ростом популярности \underline{JIT-компиляции} разделение \underline{трансляторов} на
      \underline{компиляторы и интерпретаторы} не столь актуально

      \vspace{1cm}
      Большинство \underline{актуальных интерпретаторов} стало, грубо говоря,
      \underline{компиляторами} в \underline{машинный код})

      \vspace{1cm}
      \url{https://ru.wikipedia.org/wiki/JIT}
    }

    \frame {
      Лучше разделять сами языки, а не трансляторы:
      \begin{itemize}
        \item \textbf{Компилируемые} (Assembler, C++, C\#, Java, ...)
        \item \textbf{Скриптовые} (Python, Ruby, JavaScript/ECMAScript, bash/shell, cmd/bat, PowerShell, ...)
      \end{itemize}
    }

    \frame {
      \frametitle{Компилируемые языки}
      Исходный код компилируют в \underline{двоичный код} и (обычно) распространяют в скомпилированном виде (.exe, .jar, .dll и т.д.)

      \vspace{0.5cm}
      Часто используется для:
      \begin{itemize}
        \item \textbf{прикладного программирования} (от текстового редактора до веб-браузера или более сложной системы)
        \item \textbf{системного программирования} (драйвера устройств и т.д.)
      \end{itemize}
    }

    \frame {
      \frametitle{Скриптовые языки (языки сценариев)}
      \framesubtitle{Для многих --- синоним к <<Интерпретируемым языкам>>}
      Программы обычно распространяют в виде исходного кода (.py,~.js,~.bat~и~т.д.)

      \vspace{0.5cm}
      Часто используется для:
      \begin{itemize}
        \item \textbf{прототипов} прикладных программ
        \item \textbf{сценариев} для автоматизации: сборка/тестирование билда, deploy на сервер
        \item \textbf{плагинов/расширений} (для браузера, скажем)
        \item клиентский или серверный \textbf{код для веб-сайта}
      \end{itemize}

      \vspace{0.5cm}
      Популярно там, где хочется \underline{быстро увидеть результат}
    }

    \frame {
      А еще языки классифицируют по тому, на какие \textbf{парадигмы~программирования} сделан основной акцент...
    }

    \frame {
      \frametitle{Wait, wait...}
      \framesubtitle{Зачем об этом всём знать? Для осознания того}

      \begin{itemize}
        \item что языки можно \underline{очень по-разному} классифицировать
        \item \underline{одни классы} эффективно решают \underline{одни типы задач}, другие классы --- другие типы задач
        \item{} <<универсального>> языка \underline{не существует}
        \item чем более \underline{узкоспециализирован} язык (SQL для БД, G-Codes для станков, ...) или комбинация языка и фреймворка (Ruby+RoR или JS+node.js для веб) тем \underline{быстрее} их можно изучить
      \end{itemize}

      Другими словами --- чтобы знать по какому принципу выбирать следующий язык для изучения

      \vspace{0.5cm}
      \url{https://www.youtube.com/watch?v=LR8fQiskYII}
      \url{https://www.youtube.com/watch?v=NvWTnIoQZj4}
    }

    \frame {
      \frametitle{На википедии}
      Можно увидеть разницу между языками, если обращать внимание на \underline{Paradigm} и \underline{Typing discipline}
      \begin{center}
        \includegraphics[scale=0.18]{pictures/python.png}
        \includegraphics[scale=0.2]{pictures/js.png}
        \includegraphics[scale=0.12]{pictures/java.png}
      \end{center}
    }
\end{document}
