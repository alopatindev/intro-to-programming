\documentclass{beamer}

\usepackage[utf8]{inputenc}
\usepackage[russian]{babel}

%\usepackage{beamerthemesplit}
\definecolor{myblue}{rgb}{.0,.2,.3}
\setbeamercolor*{palette primary}{use=structure,fg=white,bg=myblue}
\setbeamertemplate{navigation symbols}{}

\usepackage{tikz}
\usetikzlibrary{shapes.geometric, arrows}
\tikzstyle{io} = [trapezium, trapezium left angle=70, trapezium right angle=110, minimum width=1cm, minimum height=0.33cm, text centered, draw=black, fill=blue!30]
\tikzstyle{process} = [rectangle, minimum width=1cm, minimum height=0.31cm, text centered, draw=black, fill=orange!30]
\tikzstyle{decision} = [diamond, minimum width=1cm, minimum height=0.31cm, text centered, draw=black, fill=green!30]
\tikzstyle{arrow} = [thick,->,>=stealth]

\title{Введение в программирование}
\author{Лопатин Александр}
\date{2015}


\subtitle{Лекция 3}

\begin{document}

  \frame{\titlepage}


  \section*{Содержание} {
    \frame{\tableofcontents[hideallsubsections]}


  \section{Процедурное программирование}
    \begin{tikzpicture}
      \paradigmsProc
    \end{tikzpicture}

  \frame {
    \frametitle{Процедурное программирование (Procedural Programming)}
    \begin{itemize}
      \item программу разделяют на подпрограммы (процедуры и функции)
      \item на состояние программы влияют побочные эффекты
    \end{itemize}
  }

  \frame {
    \frametitle{Многоуровневые системы}
      \layersInRealLife
      \vspace{0.5cm}
      \textbf{Высокий} уровень просит либо \underline{свой} либо \underline{более низкий} уровень
      (а не наоборот)
    }

    \frame {
      \layersInRealLife
      \vspace{0.5cm}

      \textbf{Правильно}:

      Д просит П создать продукт

      П просит Х и М выполнить подзадачи для него

      М и Х возвращают результаты своих работ П

      П соединил результаты в продукт и вернул Д
    }

    \frame {
      \layersInRealLife
      \vspace{0.5cm}

      \textbf{Неправильно}:
      Д просит Х написать программу
    }

    \frame {
      \layersInRealLife
      \vspace{0.5cm}

      \textbf{Неправильно}:

      М вызвался сам сделать продукт. Сделал и впарил результат Д
    }

    \frame {
      \layersInRealLife
      \vspace{0.5cm}

      \textbf{Неправильно}:

      П просит Д дать ему задачу <<разработать велосипед>>
    }

    \frame {
      \layersInRealLife
      \vspace{0.5cm}

      \textbf{Плохо / спорно}:

      Д просит М выполнить задачу

      (Лучше обратиться к П, чтобы тот передал задачу М)
    }

    \frame {
      \frametitle{В процедурных программах тоже самое}

      Пример типичной архитектуры процедурной программы:
      \vspace{0.5cm}
      \begin{itemize}
        \item Главная логика
          \begin{itemize}
            \item Ввод данных
              \begin{itemize}
                \item Валидация ввода
              \end{itemize}
            \item Обработка данных
              \begin{itemize}
                \item Helpers для обработки данных
              \end{itemize}
            \item Вывод данных
          \end{itemize}
      \end{itemize}
    }

    \frame {
      \frametitle{В процедурных программах тоже самое}

      Пример типичной архитектуры процедурной программы:
      \vspace{0.5cm}
      \begin{itemize}
        \item Главная логика (main)
          \begin{itemize}
            \item Ввод данных (inputAllStudents)
              \begin{itemize}
                \item Валидация ввода (isNumber)
              \end{itemize}
            \item Обработка данных (findFailedStudents)
              \begin{itemize}
                \item Helpers для обработки данных (isFailedGrade)
              \end{itemize}
            \item Вывод данных (outputFailedStudents)
          \end{itemize}
      \end{itemize}
    }

    \frame {
      \frametitle{Пример процедурной программы (Python)}
    }

\begin{frame}[fragile]
  \begin{minted}{python}
allStudents = []
failedStudentsIndexes = []

# high-level functions (1)
def inputAllStudents(): pass        # 1
def outputFailedStudents(): pass    # 5
def findFailedStudents(): pass      # 3

# low-level functions (2)
def isNumber(s): pass               # 2
def isFailedGrade(grade): pass      # 4

if inputAllStudents():
    findFailedStudents()
    outputFailedStudents()
else:
    print "input error"
  \end{minted}
\end{frame}

\begin{frame}[fragile]
  \begin{minted}{python}
allStudents = []
...
def inputAllStudents():
    global allStudents
    nStr = raw_input("input students number: ")
    if not isNumber(nStr):
        return False
    else:
        ...
  \end{minted}
\end{frame}

\begin{frame}[fragile]
  \begin{minted}{python}
    else:
        n = int(nStr)
        for i in range(n):
            name = raw_input("input name: ")
            gradeStr = raw_input("input grade: ")
            if not isNumber(gradeStr):
                return False
            else:
                grade = int(gradeStr)
                pair = [name, grade]
                allStudents.append(pair)
        return True
  \end{minted}
\end{frame}

\begin{frame}[fragile]
  \begin{minted}{python}
def isNumber(s):
    if len(s) > 0:
        for character in s:
            if character < '0' or character > '9':
                return False
        return True
    else:
        return False
  \end{minted}
\end{frame}

\begin{frame}[fragile]
  \begin{minted}{python}
allStudents = [['Вася', 2], ['Петя', 5]]
failedStudentsIndexes = []
...
def findFailedStudents():
    global failedStudentsIndexes
    n = len(allStudents)
    for i in range(n):
        name = allStudents[i][0]
        grade = allStudents[i][1]
        if isFailedGrade(grade):
            failedStudentsIndexes.append(i)
  \end{minted}
\end{frame}

\begin{frame}[fragile]
  \begin{minted}{python}
def isFailedGrade(grade):
    if grade <= 2:
        return True
    else:
        return False
  \end{minted}
\end{frame}

\begin{frame}[fragile]
  \begin{minted}{python}
BAD_GRADE = 2
...
def isFailedGrade(grade):
    if grade <= BAD_GRADE:
        return True
    else:
        return False
  \end{minted}
\end{frame}

\begin{frame}[fragile]
  \begin{minted}{python}
BAD_GRADE = 2
...
def isFailedGrade(grade):
    return grade <= BAD_GRADE
  \end{minted}
\end{frame}

\begin{frame}[fragile]
  \begin{minted}{python}
allStudents = [['Вася', 2], ['Петя', 5]]
failedStudentsIndexes = [0]
...
def outputFailedStudents():
    for i in failedStudentsIndexes:
        name = allStudents[i][0]
        print name
  \end{minted}
\end{frame}

\begin{frame}[fragile]
  Спорный момент: вообще-то этому коду не место в самом высоком уровне:
  \begin{minted}{python}
    print "input error"
  \end{minted}
\end{frame}

\begin{frame}[fragile]
  Правильней будет выделить и его в отдельную high-level функцию:

  \vspace{0.5cm}
  \begin{minted}{python}
def printError():
    print "input error"
...
else:
    printError()
  \end{minted}

  \vspace{0.5cm}
  или вовсе вставить внутрь существующей high-level функции (inputAllStudents)
\end{frame}

\end{document}
